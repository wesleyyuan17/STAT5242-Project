\section{Methods} \label{Methods}

Traditional asset return predictions utilized time-series models that take into account the temporal nature of trading as sequential decision making. As such, most deep-learning applications in this area have leveraged the temporal aspect of RNN models, particularly LSTM, in making predictions (\cite{Shen2020}, \cite{Li2018}, \cite{Selvin2017}). Therefore, we use a vanilla LSTM model as the baseline model against which we compare our combined methods.

To improve upon the baseline, we introduce a GNN that we use to learn relational information from an implicit graph structure derived from the assets. The connectivity of the assets is evident in the log-returns correlation of the assets in our dataset as seen in figure \ref{fig:asset_corr}. This phenomenon can largely attributed to the fact that agents acting in markets tend to trade many assets at once to get better risk-adjusted returns through diversification. The interactions of these market agents with the assets determines the structure of our market graph representation. This approach is also used in \cite{Matsunaga2019}, \cite{Feng2019}, \cite{Sun2020}, \cite{Hou2021}, and \cite{Peng2021}.

\begin{figure}[H]
	\centering
	\includegraphics[width=\linewidth]{../../figures/correlation.pdf}
	\caption{Correlation of log returns calculated on minute close prices of assets. Note the high correlation among the most commonly traded assets like Bitcoin/Ethereum and Bitcoin/Binance Coin.}
	\label{fig:asset_corr}
\end{figure}

\cite{Peng2021} show empirically that using the correlation matrix as an adjacency matrix as input into GCN layers provided best performance. Therefore, we use the same, imposing a lookback window equal to the length of the sequence input to the LSTM model. Restricting this lookback window provides a basis for direct comparison as the GCN-augmented model will not have access to more information than the LSTM model and any performance boost would be a result of capturing relational information from the correlation matrix. For our specific GCN model, we implement the propogation rule taken from \cite{Kipf2017}.