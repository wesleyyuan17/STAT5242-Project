\documentclass[12pt]{article}

%% preamble
% ready for submission
\usepackage[preprint]{neurips_2021}
 
% \usepackage[margin=1in]{geometry} 
\usepackage{graphicx}
\usepackage{subcaption}
\usepackage{amsmath,amsthm,amssymb}
\usepackage{color}
\usepackage{fancyhdr}
\usepackage{float}
\usepackage{hyperref}
% \usepackage[round, authoryear]{natbib}
\bibliographystyle{abbrvnat} % aer/natbib/others

\definecolor{dkgreen}{rgb}{0,0.6,0}
\definecolor{gray}{rgb}{0.5,0.5,0.5}
\definecolor{mauve}{rgb}{0.58,0,0.82}

%% environment for coding
\usepackage{listings}
\lstset{frame=tb,
  language=Python,
  aboveskip=3mm,
  belowskip=3mm,
  showstringspaces=false,
  columns=flexible,
  basicstyle={\small\ttfamily},
  numbers=none,
  numberstyle=\tiny\color{gray},
  keywordstyle=\color{blue},
  commentstyle=\color{dkgreen},
  stringstyle=\color{mauve},
  breaklines=true,
  breakatwhitespace=true,
  tabsize=3
}
\newcommand{\code}[1]{\lstinline|#1|}

%% 
%  CUSTOM LATEX MATH COMMANDS 
%% 

%% for vectors, matrices, tensors, and random vectors 
\newcommand{\mVector}[1]
{
    \ifcat\noexpand#1\relax
        \bm{#1}
    \else
        \mathbf{#1}
    \fi
}
\newcommand{\mMatrix}[1]{\mathbf{#1}}
\newcommand{\mTensor}[1]{\bm{\mathscr{#1}}}
\newcommand{\rVector}[1]{\bm{#1}}

%% commands to have blackboard fonts for standard probability stuff (e.g. expected value, variance)
\newcommand{\pr}{\mathbb{P}}
\newcommand{\ev}{\mathbb{E}}
\newcommand{\var}{\mathbb{V}\mathrm{ar}}
\newcommand{\cov}{\mathbb{C}\mathrm{ov}}
\newcommand{\cor}{\mathbb{C}\mathrm{or}}
\newcommand{\bias}{\mathbb{B}\mathrm{ias}}
\newcommand{\R}{\mathbb{R}}
\newcommand{\I}{\mathbb{I}}

%% renewing binomial coefficient, this looks better 
\renewcommand{\binom}[2]{
    \begin{pmatrix}
        #1 \\ #2
    \end{pmatrix}
}

%% command for indicator functions, need bb font for numbers 
\usepackage{bbm}
\newcommand{\ind}{ \mathbbm{1} }

%% defining argmin and argmax functions
\DeclareMathOperator*{\argmax}{argmax}
\DeclareMathOperator*{\argmin}{argmin}

\begin{document}

%% title and authors 
\title{Predicting Future Asset Returns with GCN and LSTM} %replace X with the appropriate number
\author{%
  Wesley Yuan \\
  Department of Statistics \\
  Columbia University \\
  New York, NY, 10027 \\
  \texttt{wy2371@columbia.edu} \\
  \And
  Gurmehar Makker \\
  Department of Statistics \\
  Columbia University \\
  New York, NY, 10027 \\
  \texttt{gm2946@columbia.edu} \\
  \And
  Sierra Vo \\
  Department of Statistics \\
  Columbia University \\
  New York, NY, 10027 \\
  \texttt{tdv2104@columbia.edu} \\
  \And
  Aiden Kenny \\
  Department of Statistics \\
  Columbia University \\
  New York, NY, 10027 \\
  \texttt{apk2152@columbia.edu} \\
}

\maketitle

%% abstract 
\begin{abstract}
Placeholder
\end{abstract}

%% ----------------------------------------
%  MAIN DOCUMENT 
%% ----------------------------------------

%% introduction 
\section{Introduction}

The problem of predicting future returns given historical data for tradable assets has been extensively studied 
with many approaches having been explored. 
Traditional methods used time-series models such as ARIMA and GARCH to predict future price movements. 
Similarly, deep-learning models that can take advantage of temporal relations such as Long Short-Term Memory (LSTM) models 
have been applied to this problem with promising results. 
However, these methods fail to take into account the propagation of information through the market and the correlations of assets. 
In this aspect, Graph Convolutional Networks (GCN) has demonstrated good performance in regression problems. 
Combining these should allow for the capture and use of both intra-asset temporal and cross-asset relations 
to provide superior prediction performance.

\subsection{Related Works}

\subsection{Dataset}

%% methods 
\section{Methods} \label{Methods}

Traditional asset return predictions utilized time-series models that take into account the temporal nature of trading as sequential decision making. As such, most deep-learning applications in this area have leveraged the temporal aspect of RNN models, particularly LSTM, in making predictions (\cite{Shen2020}, \cite{Li2018}, \cite{Selvin2017}). Therefore, we use a vanilla LSTM model as the baseline model against which we compare our combined methods.

To improve upon the baseline, we introduce a GNN that we use to learn relational information from an implicit graph structure derived from the assets. The connectivity of the assets is evident in the log-returns correlation of the assets in our dataset as seen in figure \ref{fig:asset_corr}. This phenomenon can largely attributed to the fact that agents acting in markets tend to trade many assets at once to get better risk-adjusted returns through diversification. The interactions of these market agents with the assets determines the structure of our market graph representation. This approach is also used in \cite{Matsunaga2019}, \cite{Feng2019}, \cite{Sun2020}, \cite{Hou2021}, and \cite{Peng2021}.

\begin{figure}[H]
	\centering
	\includegraphics[width=\linewidth]{../../figures/correlation.pdf}
	\caption{Correlation of log returns calculated on minute close prices of assets. Note the high correlation among the most commonly traded assets like Bitcoin/Ethereum and Bitcoin/Binance Coin.}
	\label{fig:asset_corr}
\end{figure}

\cite{Peng2021} show empirically that using the correlation matrix as an adjacency matrix as input into GCN layers provided best performance. Therefore, we use the same, imposing a lookback window equal to the length of the sequence input to the LSTM model. Restricting this lookback window provides a basis for direct comparison as the GCN-augmented model will not have access to more information than the LSTM model and any performance boost would be a result of capturing relational information from the correlation matrix. For our specific GCN model, we implement the propogation rule taken from \cite{Kipf2017}.

%% results 
\section{Experiments}

We use the crypto data described above to train and test the performance of the different models. Due to computational limitations, we take the last 100k time steps with 90k time steps for training and the last 10k used as a test set. For LSTM and combined models, a sequence of 10 time-steps is taken together as inputs to the model. For GCN, a single frame and the 10-day lookback correlation matrix is used as inputs. Training hyperparameters were tuned manually to find a reasonable best-performing combination.

Our baseline model is a single-layer vanilla LSTM model with a hidden size of 14. Hidden and memory states are initialized to zero tensors. The embeddings at each of the 10 steps is flattened and fed through a final fully connected layer for predictions. Combined models use the baseline model and a single-layer GCN model that uses the above update and creates a 3-dimensional embedding that is then fed through a fully-connected layer to get the model's prediction.

\subsection{Results}

\begin{figure}[H]
	\centering
	\includegraphics[width=\linewidth]{../../figures/vanilla_lstm_sgd_training_loss.pdf}
	\caption{Training loss of baseline LSTM model}
	\label{fig:lstm_loss}
\end{figure}

\begin{table}[H]
	\centering
	\begin{tabular}{|c|c|c|c|c|}
	\hline
	Model & Optimzier & Learning Rate & Momentum & MSE \\
	\hline
	LSTM & SGD & 0.0001 & 0.9 & 0.0002 \\
	Additive & & & & \\
	Sequential & & & & \\
	\hline
	\end{tabular}
	\caption{Hyperparameters for best performing model with average MSE on validation set}
	\label{tab:results_summary}
\end{table}

%% discussion 
\section{Discussion}

Our experiments show the improved performance of combining graphical models with classic temporal models. From Table \ref{tab:results_summary} we see that both combination schemes outperform the baseline LSTM with the TGC method achieving 10x performance. This highlights the importance of models that are able to capture both temporal and relational patterns from price data in performing asset price prediction.

These results were achieved using hyperparameters tuned to the baseline LSTM training and one can assume that even better results can be achieved by the combined models with additional hyperparameter tuning. For example, the sequential model reached saturation after a single epoch, suggesting the need for a smaller initial learning rate or some learning rate annealing scheme. 

Furthermore, our experiments were constrained by limited computational resources and therefore is only meant to highlight the improvement GCN's offer rather than the performance possible. To achieve better results, one could try increasing the number of layers in both the LSTM and the GCN models as well as using bi-directional LSTM rather than vanilla LSTM to produce better embeddings.

Extensions of this line work would include applications of better price predictions to different trading strategies. Experiments can be run to test PnL or Sharpe ratio improvement given more accurate predictions in long or short term trading. The efficacy can also be studied across asset classes as previous works, including ours, have focused on assets belonging to the same class. One can try layering GCNs with a TGC model per asset class and an over-arching GCN model to combine predictions across asset classes. Given the novelty of the approach, there is still much room to explore in the use of temporal graphical models.

%% ----------------------------------------
% END OF DOCUMENT 
%% ----------------------------------------

\newpage

%% bibliography
\input{../05-bibliography/bibliography.tex}

%% appendix 
\input{../06-appendix/appendix.tex}

\end{document}

%% ----------------------------------------
%  NOTES 
%% ----------------------------------------

%  Introduction section: describe the problem you are trying to solve, in 5-7 sentences. It is
% understood that this text may change in the final report, but it doesn’t need to.
% (c) Introduction section: describe the papers that you have read and used to inform your work
% here. Create a bibliography containing these references, and include some sentences on how
% these papers are related and how you used them to build your project. If you used an online
% resource (such as a tf.agents tutorial or something similar) as a starting point, you should
% include that as a reference also.
% (d) Introduction section: describe the data you have for this problem. For example, how many
% training/validation/test samples do you have? What are the dimensionalities of the inputs and
% outputs? If an RL problem, what are the details of the states/actions/rewards?
% (e) Methods section: what is your starting point? For example, will a simple logistic regression
% get you started? What approaches already exist to solve this problem, and how difficult are
% they to implement? Describe in 3-5 sentences what first steps you have taken to start from
% something simple and move to more complex networks. This progression, as we have discussed
% in class on several occasions, is critical to empiricism and working with deep learning.
% (f) Methods section: what architectures/problem setups will you try to get you from this simple
% method to your end goal? For example, will you use dropout or batch normalization, will you
% implement a custom tf.agent, or otherwise? Note it is not necessary to make these choices
% final; this part is about showing progress.
% (g) Results section: what results do you intend to incluude, and why will they demonstrate success?
% Add a list of your intended results and what you hope they will show. If you have early results,
% you may include figures demonstrating the results (optional).
% (h) Discussion section: what are the important takeaways from your work? what problems/opportunity
% for further work (after the semester) do you see for this project? Some of this section will be
% incomplete or speculative since the project is still underway, and that is acceptable (of course
% it should be complete upon final project submission).