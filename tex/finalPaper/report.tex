\documentclass[12pt]{article}

%% preamble
% ready for submission
\usepackage[preprint]{neurips_2021}
 
% \usepackage[margin=1in]{geometry} 
\usepackage{graphicx}
\usepackage{subcaption}
\usepackage{amsmath,amsthm,amssymb}
\usepackage{color}
\usepackage{fancyhdr}
\usepackage{float}
% \usepackage[round, authoryear]{natbib}
\bibliographystyle{abbrvnat} % aer/natbib/others

\definecolor{dkgreen}{rgb}{0,0.6,0}
\definecolor{gray}{rgb}{0.5,0.5,0.5}
\definecolor{mauve}{rgb}{0.58,0,0.82}

%% environment for coding
\usepackage{listings}
\lstset{frame=tb,
  language=Python,
  aboveskip=3mm,
  belowskip=3mm,
  showstringspaces=false,
  columns=flexible,
  basicstyle={\small\ttfamily},
  numbers=none,
  numberstyle=\tiny\color{gray},
  keywordstyle=\color{blue},
  commentstyle=\color{dkgreen},
  stringstyle=\color{mauve},
  breaklines=true,
  breakatwhitespace=true,
  tabsize=3
}
\newcommand{\code}[1]{\lstinline|#1|}

%% 
%  CUSTOM LATEX MATH COMMANDS 
%% 

%% for vectors, matrices, tensors, and random vectors 
\newcommand{\mVector}[1]
{
    \ifcat\noexpand#1\relax
        \bm{#1}
    \else
        \mathbf{#1}
    \fi
}
\newcommand{\mMatrix}[1]{\mathbf{#1}}
\newcommand{\mTensor}[1]{\bm{\mathscr{#1}}}
\newcommand{\rVector}[1]{\bm{#1}}

%% commands to have blackboard fonts for standard probability stuff (e.g. expected value, variance)
\newcommand{\pr}{\mathbb{P}}
\newcommand{\ev}{\mathbb{E}}
\newcommand{\var}{\mathbb{V}\mathrm{ar}}
\newcommand{\cov}{\mathbb{C}\mathrm{ov}}
\newcommand{\cor}{\mathbb{C}\mathrm{or}}
\newcommand{\bias}{\mathbb{B}\mathrm{ias}}
\newcommand{\R}{\mathbb{R}}
\newcommand{\I}{\mathbb{I}}

%% renewing binomial coefficient, this looks better 
\renewcommand{\binom}[2]{
    \begin{pmatrix}
        #1 \\ #2
    \end{pmatrix}
}

%% command for indicator functions, need bb font for numbers 
\usepackage{bbm}
\newcommand{\ind}{ \mathbbm{1} }

%% defining argmin and argmax functions
\DeclareMathOperator*{\argmax}{argmax}
\DeclareMathOperator*{\argmin}{argmin}

\begin{document}

%% title and authors 
\title{Predicting Future Asset Returns with GCN and LSTM} %replace X with the appropriate number
\author{%
  Wesley Yuan \\
  Department of Statistics \\
  Columbia University \\
  New York, NY, 10027 \\
  \texttt{wy237@columbia.edu} \\
  \And
  Gurmeha Makker \\
  Department of Statistics \\
  Columbia University \\
  New York, NY, 10027 \\
  \texttt{gm2946@columbia.edu} \\
  \And
  Sierra Vo \\
  Department of Statistics \\
  Columbia University \\
  New York, NY, 10027 \\
  \texttt{tdv2104@columbia.edu} \\
  \And
  Aiden Kenny \\
  Department of Statistics \\
  Columbia University \\
  New York, NY, 10027 \\
  \texttt{apk2152@columbia.edu} \\
}

\maketitle

%% abstract 
\begin{abstract}
Placeholder
\end{abstract}

%% ----------------------------------------
%  MAIN DOCUMENT 
%% ----------------------------------------

%% introduction 
\section{Introduction}

The problem of predicting future returns given historical data for tradable assets has been extensively studied 
with many approaches having been explored. 
Traditional methods used time-series models such as ARIMA and GARCH to predict future price movements. 
Similarly, deep-learning models that can take advantage of temporal relations such as Long Short-Term Memory (LSTM) models 
have been applied to this problem with promising results. 
However, these methods fail to take into account the propagation of information through the market and the correlations of assets. 
In this aspect, Graph Convolutional Networks (GCN) has demonstrated good performance in regression problems. 
Combining these should allow for the capture and use of both intra-asset temporal and cross-asset relations 
to provide superior prediction performance.

\subsection{Related Works}

\subsection{Dataset}

%% methods 
\section{Methods}



%% results 
\section{Results}



%% discussion 
\section{Discussion}



%% ----------------------------------------
% END OF DOCUMENT 
%% ----------------------------------------

%% bibliography
\nocite{*}

\bibliography{report}

%% appendix 
\appendix

\section{Appendix}

\end{document}

%% ----------------------------------------
%  NOTES 
%% ----------------------------------------

%  Introduction section: describe the problem you are trying to solve, in 5-7 sentences. It is
% understood that this text may change in the final report, but it doesn’t need to.
% (c) Introduction section: describe the papers that you have read and used to inform your work
% here. Create a bibliography containing these references, and include some sentences on how
% these papers are related and how you used them to build your project. If you used an online
% resource (such as a tf.agents tutorial or something similar) as a starting point, you should
% include that as a reference also.
% (d) Introduction section: describe the data you have for this problem. For example, how many
% training/validation/test samples do you have? What are the dimensionalities of the inputs and
% outputs? If an RL problem, what are the details of the states/actions/rewards?
% (e) Methods section: what is your starting point? For example, will a simple logistic regression
% get you started? What approaches already exist to solve this problem, and how difficult are
% they to implement? Describe in 3-5 sentences what first steps you have taken to start from
% something simple and move to more complex networks. This progression, as we have discussed
% in class on several occasions, is critical to empiricism and working with deep learning.
% (f) Methods section: what architectures/problem setups will you try to get you from this simple
% method to your end goal? For example, will you use dropout or batch normalization, will you
% implement a custom tf.agent, or otherwise? Note it is not necessary to make these choices
% final; this part is about showing progress.
% (g) Results section: what results do you intend to incluude, and why will they demonstrate success?
% Add a list of your intended results and what you hope they will show. If you have early results,
% you may include figures demonstrating the results (optional).
% (h) Discussion section: what are the important takeaways from your work? what problems/opportunity
% for further work (after the semester) do you see for this project? Some of this section will be
% incomplete or speculative since the project is still underway, and that is acceptable (of course
% it should be complete upon final project submission).